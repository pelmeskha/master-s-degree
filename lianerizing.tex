\documentclass[preprint,12pt]{article}

\usepackage{amssymb}
\usepackage{amsmath}
\usepackage{float} 
\usepackage{mathrsfs}
\usepackage{nicematrix}
\usepackage{subcaption}
\usepackage[font={small,it}]{caption}
\usepackage[russian]{babel}

\DeclareMathOperator\arctanh{arctanh}
\captionsetup{justification=centering}
\newtheorem{theorem}{Statement}

\DeclareMathOperator{\sech}{sech}
\DeclareMathOperator{\csch}{csch}
\DeclareMathOperator{\sn}{sn}


\begin{document}


\section{Лианеризация НУШ}
Рассмотрим
\begin{equation} \label{eq1}
iu_{t}+a|u|^2 u+u_{xx}=0.
\end{equation}

Пусть найдено решение \(u_{0}(x,t)\). Подставим в уравнение возмущённое решение вида
\begin{equation} \label{eq2}
u(x,t)=u_{0}(x,t)+u_{1}(x,t),
\end{equation}
где \(u_{1}(x,t)\) мало. Получим
\begin{equation} \label{eq3}
iu_{0t}+iu_{1t}+u_{0xx}+u_{1xx}+a \left(\left(2 u_{0}+u_{1}\right) {| u_{1}|}^{2}+\left(u_{0}+2 u_{1}\right) {| u_{0}|}^{2}+\overline{u_{1}} u_{0}^{2}+\overline{u_{0}} u_{1}^{2}\right)=0.
\end{equation}
Пренебрегая малыми членами, получим:
\begin{equation}
iu_{0t}+iu_{1t}+u_{0xx}+u_{1xx}+a u_{0} {| u_{0}|}^{2}+2 a u_{1} {| u_{0}|}^{2}+a \overline{u_{1}} u_{0}^{2}=0.
\end{equation}
Учитывая выполнение (\ref{eq1}) для \(u_{0}(x,t)\), получим линейное уравнение для компоненты возмущения:
\begin{equation}
iu_{1t}+u_{1xx}+2 a u_{1} {| u_{0}|}^{2}+a \overline{u_{1}} u_{0}^{2}=0.
\end{equation}
Аналогично возможно лианеризовать уравнение с нелинейностью 7 степени:
\begin{equation}\label{eq3}
iu_{t}+u_{xx}+b_{1}|u|^2 u+b_{2}|u|^4 u+b_{3}|u|^6 u=0.
\end{equation}
Подставляя  (\ref{eq2}) в (\ref{eq3}), и пренебрегая малыми членами получим:
\begin{equation}
\begin{split}
iu_{0t}+iu_{1t}&+u_{0xx}+u_{1xx}+b_{3} u_{0} {| u_{0}|}^{6}+4 b_{3} u_{1} {| u_{0}|}^{6}+3 b_{3} \overline{u_{1}} u_{0}^{2} {| u_{0}|}^{4}+b_{2} u_{0} {| u_{0}|}^{4}+\\
&+3 b_{2} u_{1} {| u_{0}|}^{4}+2 b_{2} \overline{u_{1}} u_{0}^{2} {| u_{0}|}^{2}+b_{1} u_{0} {| u_{0}|}^{2}+2 b_{1} u_{1} {| u_{0}|}^{2}+b_{1} \overline{u_{1}} u_{0}^{2}=0.
\end{split}
\end{equation}
Учитывая (\ref{eq1}) для \(u_{0}(x,t)\), получим:
\begin{equation}
\begin{split}
iu_{1t}+u_{1xx}&+b_{3} u_{0} {| u_{0}|}^{6}+4 b_{3} u_{1} {| u_{0}|}^{6}+3 b_{3} \overline{u_{1}} u_{0}^{2} {| u_{0}|}^{4}+b_{2} u_{0} {| u_{0}|}^{4}+\\
&+3 b_{2} u_{1} {| u_{0}|}^{4}+2 b_{2} \overline{u_{1}} u_{0}^{2} {| u_{0}|}^{2}+2 b_{1} u_{1} {| u_{0}|}^{2}+b_{1} \overline{u_{1}} u_{0}^{2}=0.
\end{split}
\end{equation}
Или
\begin{equation}
\begin{split}
&iu_{1t}+u_{1xx}+\left(4 b_{3} {| u_{0}|}^{6}+3 b_{2} {| u_{0}|}^{4}+2 b_{1} {| u_{0}|}^{2}\right) u_{1}+\\
&+\left(3 b_{3} u_{0}^{2} {| u_{0}|}^{4}
+2 b_{2} u_{0}^{2} {| u_{0}|}^{2}+b_{1} u_{0}^{2}\right) \overline{u_{1}}+b_{3} u_{0} {| u_{0}|}^{6}+b_{2} u_{0} {| u_{0}|}^{4}=0.
\end{split}
\end{equation}

\end{document}
